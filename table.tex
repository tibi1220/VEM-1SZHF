\documentclass{standalone}

\usepackage{tikz}
\usetikzlibrary{
  calc,
  angles,
  quotes,
  backgrounds,
  patterns,
  arrows,
  arrows.meta,
  positioning,
  intersections,
  shapes.geometric,
}

\tikzset{
  dot/.style = {
      circle,
      fill=red!80!gray,
      minimum size=#1,
      draw=black,
      inner sep=0pt, outer sep=0pt,
      ultra thick,
    },
  dot/.default = 7pt,
  gdot/.style = {
      dot,
      fill=white
    },
  dim/.style = {
      latex-latex,
      draw=teal,
      ultra thick
    },
  joint/.style = {
      circle,
      draw=black,
      ultra thick,
      fill=cyan!20,
      minimum size=4mm,
    },
  square/.style = {
      regular polygon,
      regular polygon sides=4
    },
  rod/.style = {
      rectangle,
      draw=black,
      minimum height=6mm,
      minimum width=6mm,
      fill=yellow!10,
      ultra thick,
      midway,
      outer sep=0,
    },
}


\usepackage{amsmath}
\usepackage{amssymb}
\usepackage{unicode-math}

\newcommand\iu{\mathbf{j}}
\newcommand{\rvec}[1]{\mathbfit{#1}}
\newcommand{\uvec}[1]{\widehat{\mathbfit{#1}}}
\newcommand{\rmat}[1]{\mathbf{#1}}

\usepackage{luacode}
% noindent
\begin{luacode*}
  utils = require 'lua.utils' 
  variables = require "lua.variables"
\end{luacode*}
% indent

\usepackage{xargs}
\newcommandx{\silv}[3][3=]{\directlua{utils.silv("#1", "#2", "#3")}}
\newcommandx{\sivec}[4][4=]{\directlua{utils.silv1D("#1", "#2", "#3", "#4")}}
\newcommandx{\simat}[5][5=]{\directlua{utils.silv2D("#1", "#2", "#3", "#4", "#5")}}
\newcommandx{\siten}[6][6=]{\directlua{utils.silv3D("#1", "#2", "#3", "#4", "#5", "#6")}}
\newcommand{\lv}[1]{\directlua{utils.lv("#1")}}
\newcommand{\lvmat}[3]{\directlua{utils.lv2D("#1", "#2", "#3")}}
\newcommandx{\lvmth}[6][4=,5=,6=]{\directlua{utils.lv2DTD("#1", "#2", "#3", "#4", "#5", "#6")}}
% \renewcommandx{\num}[2][2=]{\directlua{utils.silv("#1", '', "#2")}}


\begin{document}
\begin{tikzpicture}[thick]
  \def\lg{8mm}
  \def\str{1.2}

  \foreach \i in {0,1,...,9}{
      \foreach \j in {0,1,...,9}{
          \coordinate (n\i\j) at (\i*\lg,-\j*\lg);
          \node[rectangle,draw,minimum width=\lg,minimum height=\lg] at (n\i\j) {};
        }
    }

  \foreach \i in {1,2,...,5} {
      \node at (-\lg*\str, -\i*\lg*2 + \lg*2) {$U_\i$};
      \node at (-\lg*\str, -\i*\lg*2 + \lg) {$V_\i$};

      \node at (\i*\lg*2 - \lg*2, \lg*\str) {$U_\i$};
      \node at (\i*\lg*2 - \lg,   \lg*\str) {$V_\i$};
    }

  % \begin{noindent}
  \begin{luacode*}
    local D = require('lua.variables').eDOF

    local colors = { 'red', 'green', 'blue', 'cyan', 'magenta', 'yellow', 'gray' }

    for i=7,1,-1 do
      local dof = D[i]

      tex.sprint(
        [[\begin{scope}[xshift=-9*\lg*\str, yshift=-]]
          .. i - 1 ..
        [[*12mm] \node[rectangle, minimum width=\lg, minimum height=\lg, draw=]]
          .. colors[i] ..
        [[,fill=]]
          .. colors[i] ..
        [[!10]{]]
          .. i ..
        [[};]]
      )

      tex.sprint(
        [[\node[text width=60mm, align=left] at (3.85,0)]] 
          ..
        [[{$\rightarrow\quad\rmat{DOF}_{]]
          .. i ..
        [[}= \begin{bmatrix}]]
      )

      for x=1,4 do
        tex.sprint(dof[x])
        if x~=4 then
          tex.sprint("&")
        end
      end

      tex.sprint [[\end{bmatrix}$};\end{scope}]]

      for x=1,4 do
        for y=1,4 do
          tex.sprint(
            [[ \node at (n]]
              .. dof[x]-1 .. dof[y]-1 ..
            [[)[rectangle, minimum width=]]
              .. (i + 3) * 0.7 ..
            [[*1mm, minimum height=]]
              .. (i + 3) * 0.7 ..
            [[*1mm, draw=]]
              .. colors[i] ..
            [[,fill=]]
              .. colors[i] ..
            [[!10]{};]]
          )
        end
      end
    end
  \end{luacode*}
  % \end{noindent}
\end{tikzpicture}
\end{document}
